\documentclass{article}
\usepackage{hyperref}

\title{The Collaboration Experiment}

\begin{document}
\maketitle
\large
\section{The problem}
The traditional way of storing documents for a class (i.e. putting the raw .tex files in a private Google Drive somewhere and uploading PDFs for the students) is simple and easy, but doesn't allow for certain things. Namely...
\begin{itemize}
\item Students cannot access the source .tex files to directly add solutions on top of the original, modifiable files.
\item Students cannot cleanly provide feedback and suggestions to help improve the materials.
\item Students cannot easily add sources, clarifications, or other useful information for their peers to see while the assignment is in progress.
\item Simultaneous editing of the documents by multiple instructors, multiple students, or a combination of the above is nearly impossible.
\end{itemize}

Especially because student-instructor collaboration is going to be so important during QEA, it seems reasonable that a better solution exists. In hopes of learning something in this space, let's run an \textit{experiment} to see what we can do better.

\section{How this works}
This document represents my  proposal (with John Geddes) to try something better. All of these files exist in two places:
\begin{itemize}
\item At \url{https://www.overleaf.com/4127684kndxvj#/12133924/}. Overleaf is a great way to easily see, work, and collaborate on \LaTeX documents. If nobody else is working simultaneously, you can also use the "Download as ZIP" button, edit it, and then upload your changes.
\item In a git repository (either \url{https://github.com/QEA-2016-experiment1/example} or \url{https://gitlab.com/QEA-2016-test1/example}) so that multiple versions can be maintained, and students can keep their own copies of the problem sets, share progress, and so forth. Let me know which of the two interfaces you like better.
% * <eric@legoaces.org> 2016-01-20T22:45:55.084Z:
%
% What do you think is nicer? See if this comment feature is useful. Just open the "Rich Text" tab above to check it out.
%
% ^.
\end{itemize}
By using the git repository as a backend, this system should scale nicely to other tools people want to use, while the Overleaf interface is designed to make simultaneous editing by lots of viewers possible.

Synchronizing between the two locations can be done automatically, and should lead to reasonable consistency.
\end{document}
